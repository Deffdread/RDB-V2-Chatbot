\documentclass[12pt, titlepage]{article}

\usepackage{booktabs}
\usepackage{bookmark}
\usepackage{tabularx}
\usepackage{hyperref}
\hypersetup{
    colorlinks,
    citecolor=black,
    filecolor=black,
    linkcolor=red,
    urlcolor=blue
}
\usepackage[round]{natbib}

\title{SE 3XA3: Test Plan\\R-DB V2}

\author{Team \#31, R-DB V2
		\\ Jason Tsui tsuij8
		\\ Hareem Arif arifh1
		\\ Student 3 name and macid
}

\date{\today}


\begin{document}

\maketitle

\pagenumbering{roman}
\tableofcontents
\listoftables
\listoffigures

\begin{table}[bp]
\caption{\bf Revision History}
\begin{tabularx}{\textwidth}{p{3cm}p{2cm}X}
\toprule {\bf Date} & {\bf Version} & {\bf Notes}\\
\midrule
Oct 25, 2018 & 1.0 & Section 1\\
Oct 26, 2018 & 1.1 & Section 4\\
\bottomrule
\end{tabularx}
\end{table}

\newpage

\pagenumbering{arabic}

\section{General Information}

\subsection{Purpose}
This document is a Test Plan document. This document will discuss and describe the testing, validation, and verification procedures to implement a discord chat bot, R-DB V2. As of document creation, the project is still undergoing major development and the test case procedures and descriptions disscussed are subject to change.

\subsection{Scope}
This project is a python based server hosted chat bot, which is able to take user input and output into the discord community server. The scope of testing for this program will cover user input, Discord API integration, and expected outputs. 

\subsection{Acronyms, Abbreviations, and Symbols}
	
\begin{table}[hbp]
\caption{\textbf{Table of Abbreviations}} \label{Table}

\begin{tabularx}{\textwidth}{p{3cm}X}
\toprule
\textbf{Abbreviation} & \textbf{Definition} \\
\midrule
API & Aplication Programming Interface\\
VoIP & Voice over Internet Protocol\\
\bottomrule
\end{tabularx}

\end{table}

\begin{table}[!htbp]
\caption{\textbf{Table of Definitions}} \label{Table}

\begin{tabularx}{\textwidth}{p{3cm}X}
\toprule
\textbf{Term} & \textbf{Definition}\\
\midrule
Discord & Freeware VoIP application. VoIP service which connects people and creates voice chat communities\\
VoIP & Methodology for communicating over the internet\\
R-DB V2 & Name of project\\
Black Box Testing & Functional testing by probing program with inputs. Testing is concered with output of program\\
White Box Testing & Structural testing by understanding how program processing occurs. Testing is concered with soundess of program\\
Pytest & Python testing framework\\
Red Discord Bot & Original program which project is based off of\\
\bottomrule
\end{tabularx}

\end{table}	

\subsection{Overview of Document}
R-DB V2 will be following the logical structure for testing purposes. Unit testing will primarily focus on handling user input and program logic. Integration testing will be devoted to integrating the bot onto the Discord API platform. System testing will be be done by observing if chat bot commands are correct. Acceptance testing will be done by a 3rd party user to evaluate non-functional requirements.  

\section{Plan}
	
\subsection{Software Description}

The software for our product consists of a Discord Bot implemented in python coding language. The environment that the software is being created in is the IDLE environment provided with python by default. The testing that will be done will also be done in the same environment in the same language. 

\subsection{Test Team}

Our group test team is comprised of all members contributing in different aspects. As the files to be implemented have been split up into the group, testing will be done both by the creator of each file of code as well as a level of testing done by other group members to ensure higher coverage. 

\subsection{Automated Testing Approach}

As our implementation is done with python we can make use of the ability to do automated testing through unit testing and integration testing. Unit testing will be done on the individual files that make up the product to ensure that the functions within each file are performing as intended. Integration testing will be done to ensure that the files are interacting with one another in a cohesive manner and performing their collaborative functionality as intended. 

\subsection{Testing Tools}

The testing tools that being intended for use consist of unittest which is the built in standard library tool that is used for testing python code. Alongside that we are considering using pytest as well which is also a complete testing tool. 

\subsection{Testing Schedule}

   \href{3XA3 Gantt Chart.pdf}{Gantt Chart showing the project schedule.}

\section{System Test Description}
	
\subsection{Tests for Functional Requirements}

\subsubsection{Area of Testing1}
		
\paragraph{Title for Test}

\begin{enumerate}

\item{test-id1\\}

Type: Functional, Dynamic, Manual, Static etc.
					
Initial State: 
					
Input: 
					
Output: 
					
How test will be performed: 
					
\item{test-id2\\}

Type: Functional, Dynamic, Manual, Static etc.
					
Initial State: 
					
Input: 
					
Output: 
					
How test will be performed: 

\end{enumerate}

\subsubsection{Area of Testing2}

...

\subsection{Tests for Nonfunctional Requirements}

\subsubsection{Area of Testing1}
		
\paragraph{Title for Test}

\begin{enumerate}

\item{test-id1\\}

Type: 
					
Initial State: 
					
Input/Condition: 
					
Output/Result: 
					
How test will be performed: 
					
\item{test-id2\\}

Type: Functional, Dynamic, Manual, Static etc.
					
Initial State: 
					
Input: 
					
Output: 
					
How test will be performed: 

\end{enumerate}

\subsubsection{Area of Testing2}

...

\subsection{Traceability Between Test Cases and Requirements}

\section{Tests for Proof of Concept}

\subsection{Integration Testing for Bot}
		
\paragraph{Integration Tests}
Testing will focus on implementing our bot and related python scripts to the Discord API and framework.

\begin{enumerate}

\item{test-Initial Function Integration\\}

Type: Manual Functional System Testing
					
Initial State: Bot is connected to community server, planning to implement a new functionality to bot
					
Input:Command to bot on Discord server
					
Output: Bot returns appropriate response or action
					
How test will be performed: Blackbox testing for initial functionalities. Will test every bot command implemented.
					
\item{test-Edge cases\\}

Type: Automated Equivilence Testing
					
Initial State: Functionional requirement has been implemented to bot with manual testing. Testing is still not thorough to guarantee correctness.
					
Input: Edge-cases
					
Output: Expected output of select command
					
How test will be performed:  Create test cases and error handling for bot. Then select edge cases for commands and create a script to input commands for testing.

\end{enumerate}

\subsection{Structural System Testing for Bot}
		
\paragraph{Structural System Tests}
Testing will focus on the structural system of the bot.

\begin{enumerate}

\item{test-Recovery\\}

Type: Manual Functional Testing
					
Initial State: Bot is assumed to be correct
					
Input: Input incorrect commands and commands with wrong parameters
					
Output: Bot handles errors and returns to ready state
					
How test will be performed: Input incorrect commands and observe if bot is able to return to normal function
					
\item{test-Stress\\}

Type: Automated Functional Testing
					
Initial State: Bot is assumed to be correct
					
Input: Bot commands
					
Output: Bot output
					
How test will be performed:  Create a script of bot commands. Run script. Check if commands output and bot returns to ready state. 

\end{enumerate}	

\section{Comparison to Existing Implementation}	

With regards to comparing our implementation to the original implementation, we intend on testing the performance comparison by running a series of identical commands on both implementations to see how they perform. As the original implementation does not come with corresponding test files, we do not have a comparison for testing choices, although we have considered the possibility of running a portion of our test plan on the original implementation to see if the overall functionality is the same. Beyond that the results of testing on our own implementation and the comparison to the original will be updated as the project continues to develop. 
				
\section{Unit Testing Plan}
		
\subsection{Unit testing of internal functions}
		
\subsection{Unit testing of output files}		


\iffalse
\bibliographystyle{plainnat}

\bibliography{SRS}

\fi

\newpage

\section{Appendix}

This is where you can place additional information.

\subsection{Symbolic Parameters}

The definition of the test cases will call for SYMBOLIC\_CONSTANTS.
Their values are defined in this section for easy maintenance.

\subsection{Usability Survey Questions?}

This is a section that would be appropriate for some teams.

\end{document}