\documentclass[12pt]{article}

\usepackage{booktabs}
\usepackage{tabularx}
\usepackage{graphicx}
\usepackage{tabto}
\usepackage{caption}

\title{SE 3XA3: Development Plan\\Red Discord Bot}

\author{Team \#31, R-DB V2
		\\ Jason Tsui tsuij8
		\\ Hareem Arif arifh1
		\\ Abdul Elrahwan elrahwaa
}


\date{}


\begin{document}

\begin{table}[hp]
\caption{Revision History} \label{TblRevisionHistory}
\begin{tabularx}{\textwidth}{llX}
\toprule
\textbf{Date} & \textbf{Developer(s)} & \textbf{Change}\\
\midrule
Sept 26,2018 & Jason Tsui & Document creation, Sections 1-4\\
Date2 & Name(s) & Description of changes\\
... & ... & ...\\
\bottomrule
\end{tabularx}
\end{table}




\newpage

\maketitle
This is a development plan document regarding the Red Discord Bot project, selected for 3XA3 Fall 2018.

\section{Team Meeting Plan}
\tab Team members are expected to meet atleast twice a week to discuss project updates and documents during 3XA3 lab times; Tuesday 4:30PM and Thursday 9:30AM in ITB 236. Members are expected to come prepared by bringing appropriate tools and be aware of the state of the project.\\

Discussions are expected to highlight key issues, distribute work, and reflect on project goals. Project decisions made, future goals planned, and estimated timelines will be recorded in a meeting minutes style log. This log wil be available and continuously updated on the gitlab repository.




\section{Team Communication Plan}

\tab In-person communication will be typically be held during team meetings unless planned otherwise. Team meetings will primarily focus on distributing tasks, planning project goals, and discussing recurring issues. On-going communication will be done via the facebook messenger application. Members are to be added to the group chat and expected to keep the notifications on. The group chat will serve as means of communication outside the classroom and primarily be used to discuss updates and urgent issues.

\section{Team Member Roles}

\tab Team members are expected to incrementally contribute to the development of the project. There will be no team leader for this project as the team size is very small and all members are to make decisions together. Below is a table of roles:

\begin{center}
  \begin{tabular}{|c|c|c|}
  \hline
  &Name & Role \\ \hline
  1& Jason Tsui & Developer, Scribe\\ 
  2& Hareem Arif & Developer\\ 
  3&Abdul Elrahwan & Developer\\ \hline
\end{tabular}
\captionof{table}{Roles}
\end{center}

\section{Git Workflow Plan}

\tab The git workflow plan will follow a centralized plan in which each member works on a remote branch and pushes their changes to the main branch. Labels will be used exclusively to tag submitted documents. Milestones will be used when future goals have been planned and estimated during the meeting minutes. 

\section{Proof of Concept Demonstration Plan}

\tab As with any project there are significant risks present. With respect to the implementation, the difficulties are that for this project we would also like to integrate encapsulation and information hiding into the implementation without disrupting the overall flow of the way that we intend to implement the project. With regards to testing, the areas of difficulty are generating a sufficient test file that provides adequate coverage to the entire program.  Another intention is that we check for non-functional requirements via a checklist method.  There are no apparent difficulties present with using the library that is required.  There is not a significant risk with portability of the project as it only requires a Python programming environment to run, and the final product will be usable on any device running Discord. 

\section{Technology}

\tab The programming language that will be used for the implementation itself will be Python. The IDE used will be IDLE, the already integrated development environment that comes with Python installation. The testing framework we have decided on is unittest, which is the Python unit testing framework. The document generation for the code for the implementation will be done using doxygen. 

\section{Coding Style}

\section{Project Schedule}

Provide a pointer to your Gantt Chart.

\section{Project Review}

\end{document}