\documentclass[12,english]{article}
\usepackage[letterpaper, portrait, margin=1in]{geometry}

\usepackage{amsmath}
\usepackage[T1]{fontenc}
\usepackage{babel}
\usepackage{textcomp}
\usepackage{titlesec}
\setcounter{secnumdepth}{4}
\usepackage{hyperref}
\usepackage{xcolor}
\usepackage{booktabs}
\usepackage{placeins}
\usepackage{multirow}
\usepackage{booktabs}
\usepackage{tabularx}
\usepackage{graphicx}
\usepackage{tabto}
\usepackage{caption}


\hypersetup{
    bookmarks=true,         % show bookmarks bar?
      colorlinks=true,       % false: boxed links; true: colored links
    linkcolor=black,          % color of internal links (change box color with linkbordercolor)
    citecolor=green,        % color of links to bibliography
    filecolor=magenta,      % color of file links
    urlcolor=cyan           % color of external links
}

\titleformat{\paragraph}
{\normalfont\normalsize\bfseries}{\theparagraph}{1em}{}
\titlespacing*{\paragraph}
{0pt}{3.25ex plus 1ex minus .2ex}{1.5ex plus .2ex}

\title{SE 3XA3: Module Internal Specification\\Red Discord Bot}
\author{Team \#31, R-DB V2
		\\ Jason Tsui tsuij8
		\\ Hareem Arif arifh1
		\\ Abdul Elrahwan elrahwaa
}


\begin{document}

\maketitle
This is an MIS for the implementation of Red Discord Bot as of 08/11/2018

\begin{table}[hp]
\caption{Revision History} \label{TblRevisionHistory}
\begin{tabularx}{\textwidth}{llX}
\toprule
\textbf{Date}  & \textbf{Change}\\
\midrule
Nov 08,2018 & Initial implementation\\
... & ... & ...\\
\bottomrule
\end{tabularx}
\end{table}


\date{}


\newpage
\tableofcontents
\newpage


\section{Module Hierarchy}
\begin{table}[!htbp]
        \begin{tabular}{ll}
        \toprule
        Level 1 & Level 2 \\
        \midrule
        General Module & \\
         \midrule
        Cogs Module & Music Module* \\
        & Images Module*\\
        & Moderation Module*\\
        & Trivia Module*\\ 
        \bottomrule
        \end{tabular}
        \caption{Module Hierarchy}
       *: Not yet implemented
        % Colour for the rulings in tables:
        \makeatletter
           \def\rulecolor#1#{\CT@arc{#1}}
           \def\CT@arc#1#2{%
           \ifdim\baselineskip=\z@\noalign\fi
           {\gdef\CT@arc@{\color#1{#2}}}}
           \let\CT@arc@\relax
          \rulecolor{black!50}
        \makeatother
        \label{Table 1}
        \end{table}
        
\section{MIS of General Module}
		\subsection{Interface Syntax}
		\subsubsection{Exported Access Programs}
		\begin{tabular}[pos]{|c|c|c|c|}
			
			\hline
			%	\label
			\textbf{Name}& \textbf{In} & \textbf{Out} & \textbf{Exceptions} \\ \hline
			OnReady & - & String & -\\ \hline
			OnMessage & String & String & -\\ \hline
			OnMessageDelete & String & String & -\\ \hline
			OnMessageEdit & String & String & -\\ \hline
			owner & - & String & -\\ \hline
			echo & Arguments & String & -\\ \hline
			clear & Context, Integer & String & -\\ 
			 \hline
			
		\end{tabular}
		
		\subsection{Interface Semantics}
		\subsubsection{State Variables}
			Token: String - Bot specific identification token\\
		ONDELETE: int - Variable for blocking onMessageDelete when command !clear is called.\\

		
		\subsubsection{Environmental Variables}
		Not Applicable
		
		\subsubsection{Assumptions}
		Token variable is a valid bot token generated from the discord website.
		
		\subsubsection{Access Program Semantics}
		
		
		onReady():
		
		Input: None
		
		Transition: None
		
		Output: Print 'Bot is ready'
		
		Exceptions: None\\
		\\
		
		OnMessage(message):
		
		Input: Message message
		
		Transition: Finds the name of the message's author and content.
		
		Output: Printed statement in the format author: message
		
		Exceptions: none\\ 
		\\
		OnMessageDelete(message):
		
		Input: Message message
		
		Transition: If the ONDELETE variable is positive, nothing happens. Otherwise it finds the message's author, content and the channel it was sent on. Then it notifies the server that the message was deleted.
		
		Output: Printed statement in the format 'Message deleted by {author}'
		
		Exceptions: none\\ 
		\\ 
		OnMessageEdit(message):
		
		Input: Message message
		
		Transition: finds the message's author, content and the channel it was sent on. Then it notifies the server when the message was edited.
		
		Output: Printed statement in the format 'Message edited by {author}'
		
		Exceptions: none\\
		\\
		owner():
		
		Input: None
		
		Transition: None
		
		Output: Print Name of the server's owner.
		
		Exceptions: none\\
		\\
		echo():
		
		Input: *args
		
		Transition: Creates a variable called output and adds all the arguments into the output variable.
		
		Output: Print content of the output variable
		
		Exceptions: none\\ 
		\\
		clear():
		
		Input: Context ctx, Int amount
		
		Transition: Access the ONDELETE variable and adds 1 to its value. It then creates an array of 'amount' number of messages in the channel and calles the delete messages function for all of the array's contents.

		Output: Prints a statement in the format 'amount' Messages cleared by admin
		
		Exceptions: none\\

\end{document}